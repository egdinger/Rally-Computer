\documentclass[letterpaper,11pt]{article}
% define the title
\author{Eric Dinger}
\title{Rally Computer Design Specifications}

%Here are some commands that I've created to use throughout this paper

%This command renders I2C correctly, it's called IIC because non letters are not allowed in macro names
\newcommand{\IIC}
{I$^2$C}

\begin{document}

% generate the title
\maketitle

% insert the table of contents
\tableofcontents
\section{Market Research}
Looking at the market, no widely used rally computers are made in America for the type of rallying done here. All the common ones are made in England. Another problem is that outside of Monit, 
no rally computers have been designed in the last decade to take advantage of advances of technology and interfaces (Alfa recently upgraded to a new display technology). In addition these are all closed source systems (even Alfa who will sell you their source), by making my rally computer open source I hope to encourage development of new features that may be too small for a manufacture to focus on, but would be useful to some. In addition as part of the long term design I would like to feature integration with smartphones either for display or dataloging.

The following is a small sample of feelings I've encountered on the internet about existing rally computers and observations I've noticed broken down by brand.
\begin{itemize}
	\item \textbf{TerraTrip}
	\begin{itemize}
		\item Unreliable
		\item Awkward interface
		\item Display can be hard to read
		\item Has clock
	\end{itemize}
	\item \textbf{Brantz (2 S Pro)}
	\begin{itemize}
		\item Simple
		\item No clock
	\end{itemize}
	\item \textbf{Coralba}
	\begin{itemize}
		\item TSD focused
		\item Expensive
		\item Programable to an extent
		\item Has clock
	\end{itemize}
	\item \textbf{Monit}
	\begin{itemize}
		\item Some version have clock
		\item Graphical LCD based display, menu based interface
		\item 2 or 4 button interface
		\item Seems popular, mainly for it's ease of use.
	\end{itemize}
\end{itemize}


\section{High Level Overview}
Looking at these points and what I want out of the first generation of my Rally computer I selected a few items
\begin{itemize}
	\item Stage rally focused
	\item 3 Odometers (1 total, 2 interval)
	\item Switchable odometer directions (on 2 odo's)
	\item Current speed display
	\item No clock
	\item Modern menu based interface
	\item Automatic factor computation
	\item Factor = number of pulses per mile or kilometer (depending on unit preference)
	\item Character (not graphic) LCD display
	\item Keybad interface
	\item Open source software and hardware
\end{itemize}

\section{Hardware overview}
\subsection{Display}
In the first prototype the display will consist of one 20x4 hitachi HD47780 based character LCD display using an adafruit \IIC \& SPI backpack. This option was chosen for several reasons, the hardware is on hand and easy/cheap to acquire if either is damaged during testing. The reason for going with the serial backpack was simple, running the display even in 4 bit mode requires many pins on the microcontroller, using the backpack requires only 4 (3 of which can be shared) when using SPI mode. SPI mode was chosen because of its high throughput compared to \IIC.

Future prototypes and production will feature a New Haven Display 20x4 OLED character display, this is the same type of display used by the refreshed Alfa rally computers. These displays are dimensionally equivalent to the 20x4 LCD display and features onboard SPI communications, eliminating the need for serial backpack. The instruction sets differ between the two displays though.

\subsection{Processor}
The design will be based on the Atmel ATMEGA 328 microcontroller. This chip is more than powerful enough for the application, has enough pins, and is throughly documented on the internet in various locations, which is useful since I want this project to be added to by rally enthusiasts, who are not necessarily programmers.

\subsection{Probe interface}
The idea is to be compatible with existing probes and stock VSS signals as possible. Currently the design uses a 2 pole Low-pass filter followed by a Schmidt trigger. The low-pass filter and Schmidt combo has a ceiling frequency of approximately 1.3KHz. The input to this filter is expected to be a 0-5v square wave.

\subsection{Power supply}
Coming soon. Currently powered off a USB car charger. Automobile power environments are a nasty place, and although I haven't ran any experiments I assume that race cars are worse due to many reasons.

\subsection{Keypad}
Currently a 4x4 matrix keypad manufactured by Greyhill.

\subsection{Misc.}
Connections to the outside world will use twist-lock connectors. Not looking at mil-spec at this time due to cost ({\raise.17ex\hbox{$\scriptstyle\mathtt{\sim}$}}\$40 per connector). 

\section{Software overview}
\subsection{Design considerations}
Since the factor will be the number of pulses per mile (or km), we can use some common tire sizes to get an idea of the range of pulses expected using a sensor that measures at the lug studs.\\
\begin{tabular}{|c|c|c|}
\hline
Tire size &  lugs & PPM\\
\hline
205/50-15 & 4 & 3496.72\\
\hline
15/62-15 & 5 & 4131.21\\
\hline
185/75-13 & 4 & 3477.73\\
\hline
\end{tabular}\\
We can see that through a range of common tire sizes the pulse count doesn't differ over to large a range, say 3000 to 5000 ppm. If we take a car traveling at 120mph with 5000ppm we see that the input frequency will be \\
\begin{math}
\frac{(120*5000)}{3600}= 166.66\mathrm{Hz}
\end{math}\\
This is well below the ceiling frequency of the hardware low-pass filter, so we run into no issues there, and so far below the processor speed (16MHz) that if the interrupt handler is anywhere near efficient there will be no issues.

\subsection{Program flow}
Since this computer is quite simple and has no external sensors/chips to configure and bring up, the program flow is fairly straight forward. The following verbal description will be replaced by a uml diagram.\\

\noindent{}Startup:\\
Load stored distances and calibration.\\
Configure the interrupts and LCD/OLED display\\
Enter main program flow\\
\\
Main loop:\\
Look for key press\\
At 10Hz rate update LCD/OLED display. Depending on screen calculate distances, speed, or show current menu\\
\\
Interrupt handler:\\
Update pulse count\\
Measure time since last pulse (For speed measurement)\\

\subsection{Keypad mapping}
Since we are using a 4x4 keypad, it will be layed out in the traditional 9 key arrangment, with the addition of some other movement and entry keys. \\
\begin{tabular}{| c | c | c | c |}
\hline
1&2&3&$\uparrow$\\
\hline
4&5&6&$\downarrow$\\
\hline
7&8&9&\\
\hline
P&0& &E\\
\hline
\end{tabular}\\

\noindent{}The use of the numbers is obvious, but the $\uparrow$ and $\downarrow$ will move the cursor between selected odo's or menu entries. E switches into entry mode, and when in entry mode saves the entered data. P switches between pages (menus).

%End of document!
\end{document}